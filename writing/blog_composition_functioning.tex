\documentclass[letterpaper, 10pt]{article}

\usepackage{titling}
\usepackage{hyperref}
\usepackage{xcolor}
\usepackage{natbib}
%~~~~~~~~~~~~~~~~~~~~~~~~~~~~~~~~~~~~~~~~~~
\begin{document}

\setlength{\droptitle}{-10em} 
\title{\textbf{A Brief History of Research on Relationships between Natural Systems' Composition and
Functioning}\vspace{-0em}}
\author{Bin Wang\vspace{-1em}}
\date{\today\vspace{-1em}}
\maketitle

Darwin wrote (\textit{Origin of Species}, 1859), \textcolor{red}{\textit{`It has been experimentally proved that if a plot of ground be sown with one species of grass, and a similar plot be sown with several distinct genera of grasses, a greater number of plants and a greater dry weight of herbage can be raised in the latter than in the former case'}}, as per \citet{harperdarwinian}. 

To my knowledge, this might be the earliest take on composition-ecosystem functioning relationship. This same question has since been being answered by generations of researchers and many more yet to come. Note that, however, a recent study stated a much shorter history of only about 25 years\citep{jochum2020results}, which I personally I do not agree with.

Even more excitingly, after generations of research it is glad to see theories of diversity-functioning are argued to be applied to human societies\citep{dunn2019biodiversity} and ecosystem restoration \citep{zhang2020improve}, among others. This short article tries to offer a brief recap of research on this topic. Specifically, a few questions are addressed as follows.

\section{What are the earliest studies on composition-functioning relationship?}
The earliest investigations on this topic can be dated back to the late 1950s \citep{de1960competitionthesis, de1960competition, harperdarwinian}. A notable study among those early works is \citep{de1960competition}, who compared different mixtures of a certain number of species. According to \citep{harperdarwinian}, the experimental models of \citep{de1960competition} are superbly designed to study the behaviour of two species in mixture and so to begin the exploration of natural diversity. In these models two species are sown together in varied proportions while the overall density of the sown or planted mixture is maintained constant. The behaviour of a species can then be compared in pure stand with its performance in variously proportioned mixtures, and the mutual aggressiveness of two forms may be measured.

However, research earlier than these exist. mutual exploitation of the light environment shown by the studies of \citet{salisbury1916oak} in woodland. However, more work in this line may be preceded by agronomic research on density-yield relationships that focused on a single species \citep{harperdarwinian} 

``Yet the rigid demonstration that mixtures of plant species outyield pure stands seems not to have been made" \citep{harperdarwinian}

Transition to studying natural diversity \citep{levin1999fragile}. Studies by Tilman et al. 


\section{What are the mechanisms of composition influencing functioning?}
Or how many mechanisms or hypotheses have been identified? Or are there any theories of diversity-functioning relationships? (as per the talk by Peter Reich \url{http://ieg4.rccc.ou.edu/seminar/video/2780a177-f3fa-4fbf-9fae-5e5f7d5c6c46.html} and a recent piece \citep{dunn2019biodiversity}). To address this grand issue, \textbf{a question should be answered in the first place is: how to quantify composition?} To a large extent, composition is implicitly assumed to be equivalent to diversity.

For example, species richness, ... However, measures matter (e.g., \citealp{schuldt2017belowground}). And the setting or systems completeness matters (\url{https://doi.org/10.2307/1939044}), which, for example, the hypothesis that complexity and stability are positively correlated was experimentally tested at the ecosystem level of organization using intact terrestrial microcosms.

Diversity decoupled from ecosystem function and resilience during mass extinction recovery:\url{https://www.nature.com/articles/s41586-019-1590-8}

\section{How can a community assemble?}
Understanding diversity-functioning relationships must address a question of how compositional heterogeneity is shaped in the first place.

\section{Function or functioning?}
\citep{jax2005function}

\section{Methodology}
\subsection{Transplanting}
The earliest transplanting study was by Theophrastus (370 BC - 285 BC)(cited in \citet{woodward1987climate}). Recent transplanting studies: Vandvik et al. 2020\citep{vandvik2020biotic}.

Transplant is not just a vegetation thing; it appears in animal. For example, Sir Peter Medawar, the farther of organ transplant, won Noble Prize for his work on graft rejection and acquired immune tolerance.

\subsection{Manipulative experiment}
A recent piece argued that experimental studies are realistic \citep{jochum2020results}.

\section{Questions not yet (well) answered}

\bibliographystyle{authordate1}
\bibliography{references}
\end{document}