\documentclass[letterpaper, 10pt]{article}

\usepackage{titling}
\usepackage{hyperref}
\usepackage{xcolor}
\usepackage{natbib}

\begin{document}

\setlength{\droptitle}{-10em} 
\title{\textbf{A Brief History of Research on Relationships between Natural Systems' Composition and
Functioning}\vspace{-0em}}
\author{Bin Wang\vspace{-1em}}
\date{\today\vspace{-1em}}
\maketitle

\textcolor{red}{Darwin wrote, `It has been experimentally proved that if a plot of ground be sown with one species of grass, and a similar plot be sown with several distinct genera of grasses, a greater number of plants and a greater dry weight of herbage can be raised in the latter than in the former case'}, as per \citet{harperdarwinian}. To my knowledge, this might be the earliest take on composition-functioning relationship. This same question has since been being answered by researchers. This article tries to offer a brief history of research on this topic. To this end, a few questions are addressed as follows.

\section{What are the earliest studies on composition-functioning relationship?}

The earliest investigations on this topic can be dated back to the late 1950s \citep{de1960competitionthesis, de1960competition, harperdarwinian}. A notable study among those early works is \citep{de1960competition}, who compared different mixtures of a certain number of species. According to \citep{harperdarwinian}, the experimental models of \citep{de1960competition} are superbly designed to study the behaviour of two species in mixture and so to begin the exploration of natural diversity. In these models two species are sown together in varied proportions while the overall density of the sown or planted mixture is maintained constant. The behaviour of a species can then be compared in pure stand with its performance in variously proportioned mixtures, and the mutual aggressiveness of two forms may be measured.

These works may be preceded by agronomic research on density-yield relationships that focused on a single species \citep{harperdarwinian} 

``Yet the rigid demonstration that mixtures of plant species outyield pure stands seems not to have been made" \citep{harperdarwinian}

Transition to studying natural diversity \citep{levin1999fragile}. Studies by Tilman et al. 


\section{What are the mechanisms of composition influencing functioning?}
Or how many mechanisms or hypotheses have been identified? Or is there a theory of diversity-functioning? (as per the talk by Peter Reich \url{http://ieg4.rccc.ou.edu/seminar/video/2780a177-f3fa-4fbf-9fae-5e5f7d5c6c46.html}). 

To address this grand issue, \textbf{a question should be answered in the first place is: how to quantify composition?} For example, species richness, ... However, measures matter (e.g., \citealp{schuldt2017belowground})


Diversity decoupled from ecosystem function and resilience during mass extinction recovery:\url{https://www.nature.com/articles/s41586-019-1590-8}


\section{How can a community assemble?}


\section{Function or functioning?}

\section{Methodology}

\subsection{Transplanting}
The earliest transplanting study was by Theophrastus (370 BC - 285 BC)(cited in \citet{woodward1987climate}). Recent
transplanting studies: Vandvik et al. 2020\citep{vandvik2020biotic}.

Transplant is not just a vegetation thing; it appears in animal. For example, Sir Peter Medawar, the farther of
organ transplant, won Noble Prize for his work on graft rejection and acquired immune tolerance.

\subsection{Manipulative experiment}


\bibliographystyle{authordate1}
\bibliography{references}
\end{document}