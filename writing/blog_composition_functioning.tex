\documentclass[letterpaper, 10pt]{article}

\usepackage{titling}
\usepackage{hyperref}
%\usepackage{xurl}
\usepackage{xcolor}
\usepackage{natbib}
%~~~~~~~~~~~~~~~~~~~~~~~~~~~~~~~~~~~~~~~~~~
\begin{document}

\setlength{\droptitle}{-10em} 
\title{\textbf{A Brief History of Research on Relationships between Composition and
Functioning in Natural Systems}\vspace{-0em}}
\author{\textbf{Bin Wang}\\(wbwenwu@gmail.com)\vspace{0em}}
\date{\today\vspace{1em}}
\maketitle

Darwin wrote in \textit{Origin of Species} (1859), \textcolor{red}{\textit{`It has been experimentally proved that if a plot of ground be sown with one species of grass, and a similar plot be sown with several distinct genera of grasses, a greater number of plants and a greater dry weight of herbage can be raised in the latter than in the former case'}} (as per \citep{harperdarwinian}). 

To my knowledge, this might be the earliest take on composition-ecosystem functioning relationship. This same question has since been being answered by generations of researchers and many more yet to come. Note that, however, a recent study stated a much shorter history of only about 25 years\citep{jochum2020results}, which I personally I do not agree with.

Even more excitingly, after generations of research we start to see theories of diversity-functioning are argued to be applied to human societies\citep{dunn2019biodiversity} and ecosystem restoration \citep{zhang2020improve}, among others. This short article tries to offer a brief recap of the research on this topic. Specifically, a few questions are addressed as follows.

\section{What are the early studies on composition-functioning relationship?}
The earliest investigations can be dated back to the late 1950s \citep{de1960competitionthesis, de1960competition}, as reviewed by \citep{harperdarwinian}. A notable study among those early works is \citep{de1960competition}, who compared different mixtures of a certain naumber of species. According to \citep{harperdarwinian}, the experimental models of \citep{de1960competition} are superbly designed to study the behaviour of two species in mixture and so to begin the exploration of natural diversity. In these models two species are sown together in varied proportions while the overall density of the sown or planted mixture is maintained constant. The behaviour of a species can then be compared in pure stand with its performance in variously proportioned mixtures, and the mutual aggressiveness of two forms may be measured.

However, even earlier research exist. Mutual exploitation of the light environment shown by the studies of \citet{salisbury1916oak} in woodland. However, more work in this line may be preceded or paralleled by agronomic research on density-yield relationships that focused on a single species \citep{harperdarwinian} 

``Yet the rigid demonstration that mixtures of plant species outyield pure stands seems not to have been made" \citep{harperdarwinian}

Transition to studying natural diversity \citep{levin1999fragile}. Studies by Tilman et al. 


\section{What are the mechanisms of composition influencing functioning?}
Or how many mechanisms, hypotheses, and/or theories have been proposed? Or are there any real theories of diversity-functioning relationships? (as per the talk by Peter Reich \url{http://ieg4.rccc.ou.edu/seminar/video/2780a177-f3fa-4fbf-9fae-5e5f7d5c6c46.html} and a recent piece \citep{dunn2019biodiversity}).
\subsection{How to quantify composition?}
To address this grand issue, \textbf{a question should be answered in the first place is: how to quantify composition?} 
To a large extent, composition is implicitly assumed to be equivalent to diversity. For example, species richness, ... However, measures matter (e.g., \citealp{schuldt2017belowground}). And the setting or systems completeness matters (\url{https://doi.org/10.2307/1939044}), which, for example, the hypothesis that complexity and stability are positively correlated was experimentally tested at the ecosystem level of organization using intact terrestrial microcosms.

Diversity decoupled from ecosystem function and resilience during mass extinction recovery:\url{https://www.nature.com/articles/s41586-019-1590-8}

\subsection{Is the concept of keystone species obsolete?}
Another paradigm in composition-functioning relationship studies is identifying keystone/indicator species. However, is the concept of keystone species obsolete? For example, Rivett, D.W., Bell, T. Abundance determines the functional role of bacterial phylotypes in complex communities. Nat Microbiol 3, 767–772 (2018). https://doi.org/10.1038/s41564-018-0180-0

\textcolor{red}{\textbf{To-read list}}:
\begin{enumerate}
\item \url{https://scitechdaily.com/two-six-many-physicists-observe-the-emergence-of-collective-behavior/}
\item Rivett, D.W., Bell, T. Abundance determines the functional role of bacterial phylotypes in complex communities. Nat Microbiol 3, 767–772 (2018). \url{https://doi.org/10.1038/s41564-018-0180-0}
\end{enumerate}


\section{How is heterogeneity in community composition shaped?}
Understanding diversity-functioning relationships must address a question of how compositional heterogeneity being shaped in the first place. Demystifying mechanisms underlying self-organization of communities has a direct benefit for informing whether manipulative experiments are realistic, as discussed below.


\section{Function or functioning?}
\citep{jax2005function}

\section{Methodology}
\subsection{Transplanting}
The earliest transplanting study was by Theophrastus (370 BC - 285 BC)(per \citet{woodward1987climate}). Recent transplanting studies: Vandvik et al. 2020\citep{vandvik2020biotic}.

Transplant is not just a vegetation thing; it appears in animal. For example, Sir Peter Medawar, the farther of organ transplant, won Noble Prize for his work on graft rejection and acquired immune tolerance.

\subsection{Manipulative experiment}
A recent piece argued that experimental studies are realistic \citep{jochum2020results}.

\section{Questions not yet (well) answered}
Overall, on why composition matters researchers are still less devoted and puzzled. This gap is primarily present in dissecting internal working mechanisms (e.g., interactions) and building their causal relationships with system-level functioning. Progress that has been made is focused on network analysis of communities.

\textcolor{red}{To-read list}:
\begin{enumerate}
\item Sanchez-Gorostiaga A, Bajić D, Osborne ML, Poyatos JF, Sanchez A (2019) High-order interactions distort the functional landscape of microbial consortia. PLoS Biol 17(12): e3000550. \url{https://doi.org/10.1371/journal.pbio.3000550}
\item Network motifs involving both competition and facilitation predict biodiversity in alpine plant communities \url{https://doi.org/10.1073/pnas.2005759118}
\end{enumerate}

\bibliographystyle{authordate1}
\bibliography{references}
\end{document}